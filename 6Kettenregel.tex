\chapter{Die Kettenregel}

In den letzten Kapiteln haben wir gelernt, wie wir Ableitungen von Funktionen berechnen können, die aus anderen Funktionen aufgebaut wurden. Diese Funktionen wurden miteinander addiert, dividiert oder multipliziert. Es gibt jedoch noch einen weiteren wichtigen Weg, wie wir Funktionen kombinieren können; Funktionen lassen sich auch verschachteln. Die \textit{Kettenregel} hilft uns, mit verschachtelten Funktionen umzugehen.

\section{Die Kettenregel}


Betrachten Sie die Funktion
\[
h(x) = (1+2x)^5.
\] 

Natürlich gibt es verschiedene Möglichkeiten, diese Funktion nach der Variablen $x$ abzuleiten. Nehmen wir jedoch an, dass $f(x) = x^5$ und $g(x) = 1+2x$, dann lässt sich die Funktion schreiben als $h(x) =f(g(x))$. Die \textit{Kettenregel} hilft uns nun, eine so verschachtelte Funktion zu berechnen, in dem wir die Ableitungen von $f(x)$ und von $g(x)$ bestimmen:

\begin{marginfigure}
\begin{tikzpicture}
	\begin{axis}[
            axis lines=none,
            clip=false,
          ]          
          \addplot [->,textColor] plot coordinates {(0,0) (-2,-4)}; %% x axis
          \addplot [->,textColor] plot coordinates {(0,0) (0,6)}; %% y axis
          \addplot [->,textColor] plot coordinates {(0,0) (6,0)}; %% g(x) axis

          \addplot [dashed, textColor] plot coordinates {(-.7,-1.4) (1.4,-1.4)};
          \addplot [dashed, textColor] plot coordinates {(1.4,-1.4) (2.1,0)};
          \addplot [dashed, textColor] plot coordinates {(2.1,0) (2.1,4.1)};
          
          \addplot [dashed, textColor] plot coordinates {(2.6,-2.6) (3.5,0)};
          \addplot [dashed, textColor] plot coordinates {(3.5,0) (3.5,4.1)};

          \addplot [dashed, very thick, textColor] plot coordinates {(1.4,-1.4) (.8,-2.6)};
          \addplot [dashed, very thick, textColor] plot coordinates {(2.1,4.1) (3.5,4.1)};

          \addplot [very thick, penColor5] plot coordinates {(.8,-2.6) (2.6,-2.6)};
          \addplot [very thick, penColor4] plot coordinates {(3.5,4.1) (3.5,5.5)};

          \addplot [very thick,penColor,domain=(0:4)] {2+x};
          \addplot [very thick,penColor2,domain=(0:4)] {-x};

          \node at (axis cs:3.5,4.8) [anchor=west,penColor4] {$f'(g(a)){\color{penColor5}g'(a)h}$};
          \node at (axis cs:1.7,-2.6) [anchor=north,penColor5] {$g'(a)h$};
          
          \addplot[color=penColor2,fill=penColor2,only marks,mark=*] coordinates{(1.4,-1.4)};  %% closed hole          
          \addplot[color=penColor,fill=penColor,only marks,mark=*] coordinates{(2.1,4.1)};  %% closed hole          

          \node at (axis cs:1,-2.1) [anchor=south,yslant=0,xslant=0,rotate=53] {$\overbrace{\hspace{.36in}}^{h}$};
          \node at (axis cs:7,0) [anchor=east] {$g(x)$};
          \node at (axis cs:0,6.7) [anchor=north] {$y$};
          \node at (axis cs:-2.15,-4) [anchor=north] {$x$};
          \node at (axis cs:-.7,-1.4) [anchor=east] {$a$};
        \end{axis}
\end{tikzpicture}
\caption{Eine geometrische interpretation der Kettenregel. Wird $a$ um eine kleine Grösse
   $h$ vergrössert, so steigt $f(g(a))$ um $f'(g(a))g'(a)h$. Also gilt 
\[
\frac{\Delta y}{\Delta x}\approx \frac{f'(g(a))g'(a)h}{h} =
f'(g(a))g'(a).
\]}
\end{marginfigure}
\begin{mainTheorem}[Kettenregel]\index{Kettenregel}\index{Ableitungsregeln!Kettenregel}
Sind $f(x)$ und $g(x)$ differenzierbar, so gilt
\[
\ddx f(g(x)) = f'(g(x))g'(x).
\]
\end{mainTheorem}

\begin{proof}
Sei $g_0$ irgend ein $x$-Wert:
\[
f'(g_0) = \lim_{h\to 0}\frac{f(g_0+h)-f(g_0)}{h}.
\]

Wir setzen $h = g-g_0$ haben dann
\begin{equation}\label{E:chainRuleLim}
f'(g_0) = \lim_{g\to g_0} \frac{f(g)-f(g_0)}{g-g_0}.
\end{equation}
Nun möchten wir eigentlich  $g=g(x+h)$ und $g_0=g(x)$ setzen;
jedoch sind wir nicht sicher, dass
\[
g(x+h) - g(x) \ne 0\qquad\text{wenn $h\ne 0$.}
\]
Um das Problem zu lösen führen wir einen ``Fehlerterm'' $E(g)$ ein, dieser gibt die Differenz zwischen der Steigung der Sekante von
$f(g_0)$ zu $f(g)$ und $f'(g_0)$ an,
\[
E(g) = \frac{f(g)-f(g_0)}{g-g_0} - f'(g_0).
\]
Insbesondere ist $E(g)(g-g_0)$ die Differenz zwischen $f(g)$ und der Tangente von $f(x)$ im Punkt $x=g$, betrachten Sie dazu die Abbildung unten (nächste Seite):

\vspace{1cm}
\begin{tikzpicture}
	\begin{axis}[
            clip=false,
            domain=0:2, range=0:6,ymax=4,ymin=0,
            axis lines =left, xlabel=$x$, ylabel=$y$,
            every axis y label/.style={at=(current axis.above origin),anchor=south},
            every axis x label/.style={at=(current axis.right of origin),anchor=west},
            xtick={1,1.666}, ytick={1,3},
            xticklabels={$g_0$,$g$}, yticklabels={$f(g_0)$,$f(g)$},
            axis on top,
          ]         
	  \addplot [textColor,dashed] plot coordinates {(1,0) (1,1)};
          \addplot [textColor,dashed] plot coordinates {(0,1) (1,1)};
          \addplot [textColor,dashed] plot coordinates {(0,3) (1.666,3)};
          \addplot [textColor,dashed] plot coordinates {(1.666,0) (1.666,1)};

          \addplot [textColor,dashed,very thick] plot coordinates {(1,1) (1.666,1)};
          \node at (axis cs:1.333,1) [anchor=north] {$\underbrace{\hspace{.75in}}_{g-g_0}$};

          \addplot [penColor4,very thick] plot coordinates {(1.666,1) (1.666,1.666)};
          \addplot [penColor5,very thick] plot coordinates {(1.666,1.666) (1.666,3)};

          \node at (axis cs:1.666,1.333) [anchor=west] {$f'(g_0)(g-g_0)$};
          \node at (axis cs:1.666,2.333) [anchor=west] {$E(g)(g-g_0)$};

          \addplot [very thick,penColor, smooth,domain=(0:7/4)] {-1/(x-2)};
          \addplot [very thick,penColor2, smooth,domain=(0:2)] {x};

          \addplot[color=penColor,fill=penColor,only marks,mark=*] coordinates{(1.666,3)};  %% closed hole          
          \addplot[color=penColor,fill=penColor,only marks,mark=*] coordinates{(1,1)};  %% closed hole          
        \end{axis}
\end{tikzpicture}
\vspace{2cm}

Daher sehen wir dass
\begin{equation}\label{E:chainRuleDer}
f(g)-f(g_0) = \left(f'(g_0) + E(g)\right)(g-g_0),
\end{equation}
und damit
\[
\frac{f(g)-f(g_0)}{g-g_0} = f'(g_0) + E(g).
\]


Kombinieren wir dies mit Gleichung~\ref{E:chainRuleLim}, erhalten wir
\begin{align*}
f'(g_0) &= \lim_{g\to g_0}\frac{f(g)-f(g_0)}{g-g_0} \\
&= \lim_{g\to g_0}f'(g_0) + E(g)\\
&= f'(g_0) + \lim_{g\to g_0} E(g),
\end{align*}
daraus folgt, dass $\lim_{g\to g_0} E(g) = 0$. Damit verschwindet dieser Fehlerterm und wir können den anfangs eingeschlagenen Weg weitergehen. Wir starten mit der Gleichung~\ref{E:chainRuleDer}, teilen beide Seiten durch $h$ und setzen
$g=g(x+h)$ und $g_0=g(x)$
\[
\frac{f(g(x+h))-f(g(x))}{h} = \left(f'(g(x)) +
E(g(x))\right)\frac{g(x+h)-g(x)}{h}.
\]
Mit dem Grenzwert für $h$ gegen $0$:
\begin{align*}
\lim_{h\to 0}\frac{f(g(x+h))-f(g(x))}{h} &= \lim_{h\to 0}\left(f'(g(x))
+ E(g(x))\right)\frac{g(x+h)-g(x)}{h}\\
&= \lim_{h\to 0}\left(f'(g(x))
+ E(g(x))\right)\lim_{h\to 0}\frac{g(x+h)-g(x)}{h}\\
&= f'(g(x))g'(x).
\end{align*}
Damit ist $\ddx f(g(x))= f'(g(x))g'(x)$.
\end{proof}

Es benötigt auch hier etwas Übung, bis Sie die Kettenregel schnell und effizient anwenden können. Die Kettenregel ist zudem die bisher komplexeste Ableitungsregel, umso wichtiger ist es, dass Sie diese richtig trainieren!
Kehren wir wieder zum Anfangsbeispiel zurück:

\begin{example}
Berechnen Sie:
\[
\ddx (1+2x)^5
\]
\end{example}

\begin{solution}
Setzen Sie $f(x) = x^5$ und $g(x) = 1+2x$, somit
\[
f'(x) = 5x^4 \qquad\text{und}\qquad g'(x) = 2.
\]
Also
\begin{align*}
\ddx (1+2x)^5 &= \ddx f(g(x))\\ 
&=f'(g(x))g'(x) \\
&= 5(1+2x)^4\cdot 2\\
&= 10(1+2x)^4.
\end{align*}
\end{solution}


Gehen wir weiter zu einem komplizierteren Fall:

\begin{example}
Berechnen Sie:
\[
\ddx \sqrt{1+\sqrt{x}}
\]
\end{example}

\begin{solution}
Setzen Sie 
$f(x)=\sqrt{x}$ und $g(x)=1+\sqrt{x}$. Somit
\[
\sqrt{1+\sqrt{x}}=f(g(f(x))) \qquad\text{und}\qquad\ddx f(g(f(x))) = f'(g(f(x)))g'(f(x))f'(x).
\]
Da 
\[
f'(x) = \frac{1}{2\sqrt{x}} \qquad\text{und}\qquad g'(x) = \frac{1}{2\sqrt{x}}
\]
folgt, dass
\[
\ddx \sqrt{1+\sqrt{x}} = \frac{1}{2\sqrt{1+\sqrt{x}}}\cdot 1\cdot  \frac{1}{2\sqrt{x}}.
\]
\end{solution}

Die gleichzeitige Verwendeung von Kettenregel, Produktregel und Potenzregel, erlaubt uns die Quotientenregel zu umgehen.


\begin{example}
Berechnen Sie:
\[
\ddx \frac{x^3}{x^2+1}
\]
\end{example}
\begin{solution}
Die lässt sich schreiben als 
\[
\ddx x^3(x^2+1)^{-1}, 
\]
wir setzen $f(x) = x^{-1}$ und $g(x) = x^2+1$. Damit ist
\[
x^3(x^2+1)^{-1} = x^3 f(g(x)) \qquad\text{und}\qquad \ddx x^3 f(g(x)) = 3x^2f(g(x))+ x^3 f'(g(x))g'(x).
\]
Da $f'(x) = \frac{-1}{x^2}$ und $g'(x) = 2x$, schreiben Sie
\[
\ddx \frac{x^3}{x^2+1} = \frac{3x^2}{x^2+1}-\frac{2x^4}{(x^2+1)^2}.
\]
\end{solution}


\begin{exercises}

\noindent 
Berechnen Sie die Ableitungen der Funktionen. Für eine zusätzliche Übung und um Ihre Resultate zu kontrollieren, leiten Sie einige dieser Aufgaben auf mehreren unterschiedlichen Wegen ab (wenn möglich)!

\twocol

\begin{exercise} $x^4-3x^3+(1/2)x^2+7x-\pi$
\begin{answer} $4x^3-9x^2+x+7$
\end{answer}\end{exercise}

\begin{exercise} $x^3-2x^2+4\sqrt{x}$
\begin{answer} $3x^2-4x+2/\sqrt{x}$
\end{answer}\end{exercise}

\begin{exercise} $(x^2+1)^3$
\begin{answer} $6(x^2+1)^2x$
\end{answer}\end{exercise}

\begin{exercise} $x\sqrt{169-x^2}$
\begin{answer} $\sqrt{169-x^2}-x^2/\sqrt{169-x^2}$
\end{answer}\end{exercise}

\begin{exercise} $(x^2-4x+5)\sqrt{25-x^2}$
\begin{answer} $ (2x-4)\sqrt{25-x^2}-$\hfill\break$(x^2-4x+5)x/\sqrt{25-x^2}$
\end{answer}\end{exercise}

\begin{exercise} $\sqrt{r^2-x^2}$, $r$ ist eine Konstante
\begin{answer} $-x/\sqrt{r^2-x^2}$
\end{answer}\end{exercise}

\begin{exercise} $\sqrt{1+x^4}$
\begin{answer} $2x^3/\sqrt{1+x^4}$
\end{answer}\end{exercise}

\begin{exercise} ${1\over\sqrt{5-\sqrt{x}}}$
\begin{answer} ${1\over 4\sqrt{x}(5-\sqrt{x})^{3/2}}$
\end{answer}\end{exercise}

\begin{exercise} $(1+3x)^2$
\begin{answer} $ 6+18x$
\end{answer}\end{exercise}

\begin{exercise} ${(x^2+x+1)\over(1-x)}$
\begin{answer} ${2 x + 1\over1 - x }+{x^2  + x + 1\over(1 - x)^2}$
\end{answer}\end{exercise}

\begin{exercise} ${\sqrt{25-x^2}\over x}$
\begin{answer} $ -1/\sqrt{25-x^2}-\sqrt{25-x^2}/x^2$
\end{answer}\end{exercise}

\begin{exercise} $\sqrt{{169\over x}-x}$
\begin{answer} ${1\over2}\left({-169\over x^2}-1\right)\Big/\sqrt{{169\over x}-x}$
\end{answer}\end{exercise}

\begin{exercise} $\sqrt{x^3-x^2-(1/x)}$
\begin{answer} $ {3x^2-2x+1/x^2\over 2\sqrt{x^3-x^2-(1/x)}}$
\end{answer}\end{exercise}

\begin{exercise} $100/(100-x^2)^{3/2}$
\begin{answer} $ {300 x \over(100-x^2)^{5/2}}$
\end{answer}\end{exercise}

\begin{exercise} ${\root 3 \of{x+x^3}}$
\begin{answer} $ { 1 + 3 x^2\over3(x+x^3)^{2/3}}$
\end{answer}\end{exercise}

\begin{exercise} $\sqrt{(x^2+1)^2+\sqrt{1+(x^2+1)^2}}$
\begin{answer} $ \left(4x(x^2+1)+{4x^3+4x\over2\sqrt{1+(x^2+1)^2}}\right)\Big/$
\hfill\break$2\sqrt{(x^2+1)^2+\sqrt{1+(x^2+1)^2}}$
\end{answer}\end{exercise}

\begin{exercise} $(x+8)^5$
\begin{answer} $5(x+8)^4$
\end{answer}\end{exercise}

\begin{exercise} $(4-x)^3$
\begin{answer} $-3(4-x)^2$
\end{answer}\end{exercise}

\begin{exercise} $(x^2+5)^3$
\begin{answer} $6x(x^2+5)^2$
\end{answer}\end{exercise}

\begin{exercise} $(6-2x^2)^3$
\begin{answer} $-12x(6-2x^2)^2$
\end{answer}\end{exercise}

\begin{exercise} $(1-4x^3)^{-2}$
\begin{answer} $24x^2(1-4x^3)^{-3}$
\end{answer}\end{exercise}

\begin{exercise} $5(x+1-1/x)$
\begin{answer} $5+5/x^2$
\end{answer}\end{exercise}

\begin{exercise} $4(2x^2-x+3)^{-2}$
\begin{answer} $-8(4x-1)(2x^2-x+3)^{-3}$
\end{answer}\end{exercise}

\begin{exercise} ${1\over 1+1/x}$
\begin{answer} $1/(x+1)^2$
\end{answer}\end{exercise}

\begin{exercise} ${-3\over 4x^2-2x+1}$
\begin{answer} $3(8x-2)/(4x^2-2x+1)^2$
\end{answer}\end{exercise}

\begin{exercise} $(x^2+1)(5-2x)/2$
\begin{answer} $-3x^2+5x-1$
\end{answer}\end{exercise}

\begin{exercise} $(3x^2+1)(2x-4)^3$
%\begin{answer} $120x^4-576x^3+888x^2-480x+96$
\begin{answer} $6x(2x-4)^3+6(3x^2+1)(2x-4)^2$
\end{answer}\end{exercise}

\begin{exercise} ${x+1\over x-1}$
\begin{answer} $-2/(x-1)^2$
\end{answer}\end{exercise}

\begin{exercise} ${x^2-1\over x^2+1}$
\begin{answer} $4x/(x^2+1)^2$
\end{answer}\end{exercise}

\begin{exercise} ${(x-1)(x-2)\over x-3}$
\begin{answer} $(x^2-6x+7)/(x-3)^2$
\end{answer}\end{exercise}

\begin{exercise} ${2x^{-1}-x^{-2}\over 3x^{-1}-4x^{-2}}$
\begin{answer} $-5/(3x-4)^2$
\end{answer}\end{exercise}

\begin{exercise} $3(x^2+1)(2x^2-1)(2x+3)$
\begin{answer} $60x^4+72x^3+18x^2+18x-6$
\end{answer}\end{exercise}

\begin{exercise} ${1\over (2x+1)(x-3)}$
\begin{answer} $(5-4x)/((2x+1)^2(x-3)^2)$
\end{answer}\end{exercise}

\begin{exercise} $((2x+1)^{-1}+3)^{-1}$
\begin{answer} $1/(2(2+3x)^2)$
\end{answer}\end{exercise}

\begin{exercise} $(2x+1)^3(x^2+1)^2$
\begin{answer} $56x^6+72x^5+110x^4+100x^3+60x^2+28x+6$
\end{answer}\end{exercise}

\endtwocol

\begin{exercise}  Finden Sie eine Gleichung für die Tangente an 
$f(x) = (x-2)^{1/3}/(x^3 + 4x - 1)^2$ im Punkt  $x=1$.
\begin{answer} $y=23x/96-29/96$
\end{answer}\end{exercise}

\begin{exercise} Finden Sie eine Gleichung für die Tangente an  $y=9x^{-2}$ im Punkt $(3,1)$.
\begin{answer} $y=3-2x/3$
\end{answer}\end{exercise}

\begin{exercise} Finden Sie eine Gleichung für die Tangente an  $(x^2-4x+5)\sqrt{25-x^2}$ 
im Punkt $(3,8)$.
\begin{answer} $y=13x/2-23/2$
\end{answer}\end{exercise}

\begin{exercise} Finden Sie eine Gleichung für die Tangente an  ${(x^2+x+1)\over(1-x)}$ 
im Punkt $(2,-7)$.
\begin{answer} $y=2x-11$
\end{answer}\end{exercise}

\begin{exercise} Finden Sie eine Gleichung für die Tangente an  
$\sqrt{(x^2+1)^2+\sqrt{1+(x^2+1)^2}}$
im Punkt $(1,\sqrt{4+\sqrt{5}})$.
\begin{answer} $y=
{20+2\sqrt5\over5\sqrt{4+\sqrt5}}\,x+{3\sqrt5\over5\sqrt{4+\sqrt5}}$
\end{answer}\end{exercise}

\end{exercises}






















