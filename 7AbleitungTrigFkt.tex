\chapter{Die Ableitungen der Winkelfunktionen}



\section{Die Ableitungen der Winkelfunktionen}

Bis zu diesem Kapitel haben wir eine grosse Klasse von Funktionen ignoriert --- die Klasse der Winkelfunktionen ($\cos(x)$ und $\sin(x)$). Es wird nun Zeit, sich diesen alten Bekannten anzunehmen!


\begin{theorem}[Die Ableitung von sin(\textit{x})]\index{Ableitung!der Sinusfunktion}\label{theorem:deriv sin}
\[
\ddx \sin(x) = \cos(x).
\]
\end{theorem}
\marginnote{
\begin{align*}
\lim_{h\to 0}\frac{\cos(h)-1}{h} &= \lim_{h\to 0}\left(\frac{\cos(h)-1}{h}\cdot\frac{\cos(h)+1}{\cos(h)+1}\right)\\
&=\lim_{h\to 0}\frac{\cos^2(h)-1}{h(\cos(h)+1)}\\
&=\lim_{h\to 0}\frac{-\sin^2(h)}{h(\cos(h)+1)}\\
&=-\lim_{h\to 0}\left(\frac{\sin(h)}{h}\cdot\frac{\sin(h)}{(\cos(h)+1)}\right)\\
&= -1 \cdot \frac{0}{2} = 0.
\end{align*}
}
\begin{proof}
Wir führen den Beweis mit Hilfe der Definition der Ableitung:
\begin{align*}
\ddx \sin(x) &= \lim_{h\to0} \frac{\sin(x+h)-\sin(x)}{h} \\
&= \lim_{h\to0} \frac{\sin(x)\cos(h)+\sin(h)\cos(x)-\sin(x)}{h}  & \text{trigonometrische Identität.}\\
&= \lim_{h\to0} \left(\frac{\sin(x)\cos(h)-\sin(x)}{h} + \frac{\sin(h)\cos{x}}{h} \right)\\
&=\lim_{h\to0} \left(\sin (x)\frac{\cos(h) - 1}{h}+\cos(x)\frac{\sin(h)}{h}\right) \\
&=\sin(x) \cdot 0 + \cos(x) \cdot 1 = \cos x. & 
\end{align*}
\end{proof}

Betrachten Sie die folgende geometrische Interpretation der Ableitung der Funktion
$\sin(\theta)$.  
%\begin{figure}

\begin{tikzpicture}
	\begin{axis}[
            xmin=-.1,xmax=1.1,ymin=-.1,ymax=1.1,
            axis lines=center,
            ticks=none,
            width=5in,
            unit vector ratio*=1 1 1,
            xlabel=$x$, ylabel=$y$,
            every axis y label/.style={at=(current axis.above origin),anchor=south},
            every axis x label/.style={at=(current axis.right of origin),anchor=west},
          ]        
          \addplot [very thick, textColor!30!background, smooth, domain=(-.2:.2+pi/2)] ({cos(deg(x))},{sin(deg(x))});
          \addplot [textColor,very thick] plot coordinates {(0,0) (.766,.643)}; %% 40 degrees
          \addplot [textColor,very thick] plot coordinates {(0,0) (.766,0)}; %% bottom
          \addplot [very thick, penColor2!30!background] {(x-.766)*(-.766/.643)+.643};
          \addplot [textColor,dashed] plot coordinates {(0,0) (.766-.196,.643+1-.766)}; %% 40+16.98 degrees          

          %% \addplot [textColor!20!background] plot coordinates {(.766,.643) (1,.839)}; %% hyp
          %% \addplot [textColor!20!background] plot coordinates {(1,.643) (1,.839)}; %% side
          %% \addplot [textColor!20!background] plot coordinates {(.766,.643) (1,.643)}; %% bottom
          %% \addplot [textColor!20!background,smooth, domain=(0:40)] ({.05*cos(x)+.766},{.05*sin(x)+.643}); %% angle
          %% \node at (axis cs:.84,.670) [textColor!20!background] {\footnotesize$\theta$};
          
          %% \addplot [textColor!20!background] plot coordinates {(.766,.643) (.766,.839)}; %% side
          %% \addplot [textColor!20!background] plot coordinates {(.766,.839) (1,.839)}; %% bottom
          %% \addplot [textColor!20!background,smooth, domain=(180:220)] ({.05*cos(x)+1},{.05*sin(x)+.839}); %% angle
          %% \node at (axis cs:.926,.812) [textColor!20!background] {\footnotesize$\theta$};
          
          \draw[rotate around={30:(.5,.5)}] (.7,.7) rectangle (.25,.25);

          %\draw[textColor, rotate around={45:(.5,.5)}] (.5,.5) rectangle (.2,.2);

          \addplot [penColor4,very thick] plot coordinates {(.766,.643) (.766,.643+1-.766)}; %% side
          \addplot [textColor,very thick] plot coordinates {(.766,.643+1-.766) (.766-.196,.643+1-.766)}; %% top
          \addplot [textColor,smooth, domain=(90:130)] ({.05*cos(x)+.766},{.05*sin(x)+.643}); %% angle
          \addplot [very thick, textColor] plot coordinates {(.766-.196,.643+1-.766) (.766,.643)}; %% hyp
          \node at (axis cs:.739,.717) [textColor] {\footnotesize$\theta$};
          
          \node at (axis cs:.668,.877) [anchor=south] {\footnotesize$h\sin(\theta)$};
          \node at (axis cs:.766,.76) [anchor=west] {\footnotesize$h\cos(\theta)$};
          \node at (axis cs:.65,.78) [anchor=west] {\footnotesize$\approx h$};

          \addplot [very thick, penColor] plot coordinates {(.766,0) (.766,.643)}; %% sin theta          
          
          \addplot [textColor, smooth, domain=(0:40)] ({.15*cos(x)},{.15*sin(x)});
          \addplot [textColor, smooth, domain=(40:56.90)] ({.17*cos(x)},{.17*sin(x)});
          \addplot [textColor, smooth, domain=(40:56.90)] ({.185*cos(x)},{.185*sin(x)});
          \node at (axis cs:.15,.07) [anchor=west] {$\theta$};
          \node at (axis cs:.15,.17) {$h$};
          \node at (axis cs:.766,.322) [anchor=east] {$\sin(\theta)$};
          \node at (axis cs:.383,0) [anchor=north] {$\cos(\theta)$};
        \end{axis}
\end{tikzpicture}
%\label{figure:geo-interp sinx/x}
%\end{figure}

Wenn der Winkel $\theta$ um einen ``kleinen Wert'' $h$,
vergrössert wird, so vergrössert sich $\sin(\theta)$ um ungefähr $h\cos(\theta)$. Also,
\[
\frac{\Delta y}{\Delta \theta}\approx \frac{h\cos(\theta)}{h} =
\cos(\theta).
\]
Die Ebleitung einer Funktion beschreibt ja bekanntlich die Steigung des Graphen dieser Funktion. Wenn wir die Sinusfunktion der Kosinusfunktion gegenüberstellen, so sollte letztere exakt der Steigung der Sinusfunktion entsprechen, was tatsächlich so ist, vgl. dazu Abbildung~\ref{figure:sin/cos}.
\begin{figure*}
\begin{tikzpicture}
	\begin{axis}[
            xmin=-6.75,xmax=6.75,ymin=-1.5,ymax=1.5,
            axis lines=center,
            xtick={-6.28, -4.71, -3.14, -1.57, 0, 1.57, 3.142, 4.71, 6.28},
            xticklabels={$-2\pi$,$-3\pi/2$,$-\pi$, $-\pi/2$, $0$, $\pi/2$, $\pi$, $3\pi/2$, $2\pi$},
            ytick={-1,1},
            %ticks=none,
            width=9in,
            height=2in,
            unit vector ratio*=1 1 1,
            xlabel=$x$, ylabel=$y$,
            every axis y label/.style={at=(current axis.above origin),anchor=south},
            every axis x label/.style={at=(current axis.right of origin),anchor=west},
          ]        
          \addplot [very thick, penColor, samples=100,smooth, domain=(-6.75:6.75)] {sin(deg(x))};
          \addplot [very thick, penColor2, samples=100,smooth, domain=(-6.75:6.75)] {cos(deg(x))};
          \node at (axis cs:3.14,.75) [penColor] {$f(x)$};
          \node at (axis cs:-1.57,.75) [penColor2] {$f'(x)$};
        \end{axis}
\end{tikzpicture}
\caption{Hier sehen wir den Plot von $f(x)=\sin(x)$ und der Ableitung
  $f'(x)=\cos(x)$. Man erkennt deutlich, dass $\cos(x)$ positiv ist, solange $\sin(x)$ ansteigt und dass $\cos(x)$ negativ wird, sobald $\sin(x)$ abfällt.}
\label{figure:sin/cos}
\end{figure*}

Nun, da wir die Ableitung der Sinusfunktion kennen, ist es uns möglich auch Ableitungen komplizierterer Funktionen, die die Sinusfunktion beinhalten zu berechnen.

%\break

\begin{theorem}[Die Ableitung von cos(\textit{x})]\index{Ableitung!der Kosinusfunktion}
\[
\ddx \cos(x) = -\sin(x).
\]
\end{theorem}

\begin{proof}
Es gilt (Phasenverschiebung)
\begin{align*}
\cos(x) &= \sin\left(x+\frac{\pi}{2}\right), \\
\sin(x) &= -\cos\left(x+\frac{\pi}{2}\right).
\end{align*}
Also:
\begin{align*}
\ddx \cos(x) &= \ddx \sin\left(x+\frac{\pi}{2}\right)\\
&=\cos\left(x+\frac{\pi}{2}\right)\cdot 1 \\
&= -\sin(x).
\end{align*}
\end{proof}

Als Nächstes:

\begin{theorem}[Die Ableitung von tan(\textit{x})]\index{Ableitung!der Tangensfunktion}
\[
\ddx \tan(x) \frac{1}{\cos^2(x)} = \sec^2(x).
\]
\end{theorem}

\begin{proof}
Wir schreiben $\tan(x)$ als $\frac{\sin(x)}{\cos(x)}$ und benutzen die Quotientenregel:
\begin{align*}
\ddx\tan(x) &= \ddx\frac{\sin(x)}{\cos(x)}\\
&=\frac{\cos^2(x) + \sin^2(x)}{\cos^2(x)}\\
&=\frac{1}{\cos^2(x)}\\
&=\sec^2(x).
\end{align*}
\end{proof}

Zuletzt noch:

\begin{theorem}[Die Ableitung von sec(\textit{x})]\index{Ableitung!der Sekansfunktion}
\[
\ddx \sec(x) = \sec(x)\tan(x).
\]
\end{theorem}

\begin{proof}
Wir schreiben $\sec(x)$ als $(\cos(x))^{-1}$ und benutzen die Produktregel und die Kettenregel:
\begin{align*}
\ddx \sec(x) &= \ddx(\cos (x))^{-1}\\
&=-1(\cos(x))^{-2}(-\sin(x)) \\
&= \frac{\sin(x)}{\cos^2(x)} \\
&= \sec(x)\tan(x).
\end{align*}
\end{proof}


Somit haben wir die folgenden Ableitungsregeln für die trigonometrischen Funktionen:

\begin{mainTheorem}[Die Ableitungsregeln der trigonometrischen Funktionen] \hfil
\begin{itemize}
\item $\ddx \sin(x) = \cos(x)$.
\item $\ddx \cos(x) = -\sin(x)$.
\item $\ddx \tan(x) = \sec^2(x)$.
\item $\ddx \sec(x) = \sec(x)\tan(x)$.
\item $\ddx \csc(x) = -\csc(x)\cot(x)$.
\item $\ddx \cot(x) = -\csc^2(x)$.
\end{itemize}
\end{mainTheorem}


\begin{warning}
Wenn Sie mit Ableitungen der trigonometrischen Funktionen arbeiten, wird empfohlen die Winkel in \textbf{Radiant} zu messen!
\end{warning}



\begin{exercises}
Leiten Sie die folgenden Funktionen ab.

\twocol

\begin{exercise} $\sin^2(\sqrt{x})$
\begin{answer} $\sin(\sqrt{x})\cos(\sqrt{x})/\sqrt{x}$
\end{answer}\end{exercise}

\begin{exercise} $\sqrt{x}\sin(x)$
\begin{answer} ${\sin(x)\over2\sqrt x}+\sqrt{x}\cos(x)$
\end{answer}\end{exercise}

\begin{exercise} ${1\over \sin(x)}$
\begin{answer} $ -{\cos(x)\over \sin^2(x)}$
\end{answer}\end{exercise}

\begin{exercise} ${x^2+x\over \sin(x)}$
\begin{answer} ${(2x +1)\sin(x) - (x^2+x)\cos(x) \over \sin^2 (x)}$
\end{answer}\end{exercise}

\begin{exercise} $\sqrt{1-\sin^2(x)}$
\begin{answer} ${-\sin(x)\cos(x)\over \sqrt{1-\sin^2(x)}}$
\end{answer}\end{exercise}

\begin{exercise} $\sin (x)\cos(x)$
\begin{answer} $\cos^2(x)-\sin^2(x)$
\end{answer}\end{exercise}

\begin{exercise} $\sin(\cos(x))$
\begin{answer} $-\sin(x)\cos(\cos(x))$
\end{answer}\end{exercise}

\begin{exercise} $\sqrt{x\tan(x)}$
\begin{answer} ${\tan(x)+x\sec^2(x)\over2\sqrt{x\tan(x)}}$
\end{answer}\end{exercise}

\begin{exercise} $\tan(x)/(1+\sin(x))$
\begin{answer} ${\sec^2(x)(1+\sin(x))-\tan(x) \cos(x)\over (1+\sin(x))^2}$
\end{answer}\end{exercise}

\begin{exercise} $\cot(x)$
\begin{answer} $ -\csc^2(x)$
\end{answer}\end{exercise}

\begin{exercise} $\csc(x)$
\begin{answer} $ -\csc(x)\cot(x)$
\end{answer}\end{exercise}

\begin{exercise} $x^3 \sin (23x^2 )$
\begin{answer} $3x^2\sin(23x^2)+46x^4\cos(23x^2)$
\end{answer}\end{exercise}

\begin{exercise} $\sin ^2(x) + \cos ^2(x)$
 \begin{answer} $0$
\end{answer}\end{exercise}

\begin{exercise}  $\sin (\cos (6x) )$
 \begin{answer} $-6\cos(\cos(6x))\sin(6x)$
\end{answer}\end{exercise}

\endtwocol

\begin{exercise} Berechnen Sie ${d\over d\theta}{\sec(\theta)\over 1+\sec(\theta)}$.
 \begin{answer} $\sin(\theta)/(\cos(\theta)+1)^2$
\end{answer}\end{exercise}

\begin{exercise} Berechnen Sie ${d\over dt}t^5 \cos (6t)$.
\begin{answer} $5t^4\cos(6t)-6t^5\sin(6t)$
\end{answer}\end{exercise}

\begin{exercise} Berechnen Sie ${d\over dt}{t^3 \sin (3t)\over\cos (2t)}$.
\begin{answer} $3t^2(\sin(3t)+t\cos(3t))/\cos(2t)+2t^3\sin(3t)\sin(2t)/\cos^2(2t)$
\end{answer}\end{exercise}

\begin{exercise} Finden Sie alle Punkte auf dem Graph von
$f(x)=\sin^2(x)$ an denen die Tangente horizontal verläuft
\begin{answer} $n\pi/2$, für alle ganzen Zahlen $n$
\end{answer}\end{exercise}

\begin{exercise} Finden Sie alle Punkte auf dem Graph von $f(x) = 2\sin(x) -
\sin^2(x)$ an denen die Tangente horizontal verläuft.
\begin{answer} $\pi/2+n\pi$, für alle ganzen Zahlen $n$
\end{answer}\end{exercise}

\begin{exercise} Finden Sie eine Gleichung für die Tangente an $\sin^2(x)$ im Punkt 
$x=\pi/3$.
\begin{answer} $\sqrt3x/2+3/4-\sqrt3\pi/6$
\end{answer}\end{exercise}

\begin{exercise} Finden Sie eine Gleichung für die Tangente an $\sec^2(x)$
im Punkt  $x=\pi/3$.
\begin{answer} $8\sqrt3x+4-8\sqrt3\pi/3$
\end{answer}\end{exercise}

\begin{exercise} Finden Sie eine Gleichung für die Tangente an $\cos^2(x) -
\sin^2(4x)$ im Punkt  $x=\pi/6$.
\begin{answer} $3\sqrt3x/2-\sqrt3\pi/4$
\end{answer}\end{exercise}

\begin{exercise} Finden Sie die Punkte auf der Kurve $y= x+ 2\cos(x)$, die eine horizontale Tangente besitzen.
\begin{answer} $\pi/6+2n\pi$, $5\pi/6+2n\pi$, für alle ganzen Zahlen $n$
\end{answer}\end{exercise}

\end{exercises}





