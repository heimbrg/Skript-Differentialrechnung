\chapter{Techniques of Integration}


\section{Integration by Substitution}

Computing antiderivatives is not as easy as computing derivatives. One
issue is that the chain rule can be difficult to ``undo.'' Sometimes 
it is helpful to transform the integral in question via substitution. 

\marginnote[.5in]{Here as is customary in calculus courses, we are abusing
  notation slightly, allowing $u$ to both be a name of a function
  $u(x)$, and a variable in the second integral.}
\begin{mainTheorem}[Integral Substitution Formula] 
If $u(x)$ is differentiable on the interval $[a,b]$ and $f(x)$ is
differentiable on the interval $[u(a),u(b)]$, then
\[
\int_a^b f'(u(x)) u'(x) \d x =\int_{u(a)}^{u(b)} f'(u) \d u.
\]
\end{mainTheorem}
\begin{proof} First we recognize the chain rule
\[
\int_a^b f'(u(x)) u'(x) \d x = \int_a^b (f\circ u)'(x) \d x.
\]
Next we apply the Fundamental Theorem of Calculus. 
\begin{align*} 
\int_a^b (f\circ u)'(x) \d x &= f(u(x)) \bigg|_a^b \\
&= f(x) \bigg|_{u(a)}^{u(b)}\\ 
&= \int_{g(a)}^{g(b)} f'(u) \d u.
\end{align*}
\end{proof}


There are several different ways to think about substitution. The
first is using the formula given above. Let's see an example. 
\begin{example}
Compute
\[
\int_1^3 x\cos(x^2)\d x.
\]
\end{example}

\marginnote[1.25in]{Here we are directly using the equation
\[
\int_a^b f'(u(x)) u'(x) \d x = \int_{u(a)}^{u(b)} f'(u) \d u.
\]}

\begin{solution}
A little thought reveals that if $x\cos(x^2)$ is the derivative of
some function, then it must have come from an application of the chain
rule. Here we have $x$ on the ``outside,'' which is the derivative of
$x^2$ on the ``inside,'' 
\[
\int \underbrace{x}_{\text{outside}}\cos(\underbrace{x^2}_{\text{inside}})\d x.
\]
Set $u(x) = x^2$ so $u'(x) = 2x$ and now it must be that $f(u) =
\frac{\cos(u)}{2}$. Now we see
\begin{align*}
\int_1^3 x\cos(x^2)\d x &= \int_1^9 \frac{\cos(u)}{2}\d u\\
&= \frac{\sin(u)}{2} \bigg|_1^9 \\
&= \frac{\sin(9) -\sin(1)}{2}.
\end{align*}
\end{solution}

Sometimes we frame the solution in a different way. Let's do the same
example again, this time we'll think in terms of differentials.

\begin{example}
Compute
\[
\int_1^3 x\cos(x^2)\d x.
\]
\end{example}
\begin{solution}
Here we will set $u=x^2$. Now $du = 2x \d x$, we are thinking in terms
of differentials. Now we see
\[
\int_{u(1)}^{u(3)} \frac{\cos(u)}{2}\d u = \int_1^3\frac{\cos(x^2)}{2}2x\d x.
\]
At this point, we can continue as we did before and write
\[
\int_1^3 x\cos(x^2)\d x= \frac{\sin(9) -\sin(1)}{2}.
\]
\end{solution}

Finally, sometimes we simply want to deal with the antiderivative on
its own, we'll repeat the example one more time demonstrating this.

\begin{example}
Compute
\[
\int_1^3 x\cos(x^2)\d x.
\]
\end{example}
\begin{solution}
Here we start as we did before, setting $u=x^2$. Now $du = 2x \d x$,
again thinking in terms of differentials. Now we see
\[
\int  \frac{\cos(u)}{2}\d u = \int \frac{\cos(x^2)}{2}2x\d x.
\]
Hence 
\[
\int x\cos(x^2)\d x = \frac{\sin(u)}{2} = \frac{\sin(x^2)}{2}.
\]
Now we see
\begin{align*}
\int_1^3 x\cos(x^2)\d x &=\frac{\sin(x^2)}{2}\bigg|_1^3\\
&= \frac{\sin(9) -\sin(1)}{2}.
\end{align*}
\end{solution}

With some experience, it is not hard to see which function is $f(x)$
and which is $u(x)$, let's see another example.
\begin{example}
Compute
\[
\int x^4(x^5+1)^{99} \d x.
\]
\end{example}

\begin{solution}
Here we set $u = x^5+1$ so $du = 5x^4 \d x$, and $f(u) = \frac{u^{99}}{5}$. Now
\begin{align*}
\int x^4(x^5+1)^{99} \d x &= \int \frac{u^{99}}{5} \d u\\
&= \frac{u^{100}}{500}.
\end{align*}
Recalling that $u = x^5+1$, we have our final answer
\[
\int x^4(x^5+1)^{99} \d x= \frac{(x^5+1)^{100}}{500}+C.
\]
\end{solution}


Our next example is a bit different.

\begin{example}
Compute
\[
\int_{2}^{3} \frac{1}{x\ln(x)} \d x.
\]
\end{example}

\begin{solution}
Let $u=\ln(x)$ so $du=\frac{1}{x}\d x$. Write
\begin{align*}
\int_{2}^{3} \frac{1}{x\ln(x)} \d x = \int_{\ln(2)}^{\ln(3)} \frac{1}{u} \d u\\
&= \ln(u) \bigg|_{\ln(2)}^{\ln(3)}\\
& = \ln(\ln(3)) - \ln(\ln(2)).
\end{align*}
\end{solution}


On the other hand our next example is much harder.

\begin{example} Compute
\[
\int x^3\sqrt{1-x^2}\d x.
\]
\end{example}

\begin{solution} 
Here it is not apparent that the chain rule is involved. However, if
it was involved, perhaps a good guess for $u$ would be
\[
u = 1-x^2
\]
in this case
\[
du = -2x \d x.
\]
Now consider our indefinite integral
\[
\int x^3\sqrt{1-x^2}\d x,
\]
immediately we can substitute. Write
\[
\int x^3\sqrt{1-x^2}\d x = \int -\frac{x^2\sqrt{u}}{2}\d u.
\]
However, we cannot continue until each $x$ is replaced. We know however that 
\begin{align*}
u &= 1-x^2 \\
u -1 &= -x^2\\
1- u &= x^2
\end{align*}
so now we may write
\[
\int x^3\sqrt{1-x^2}\d x = \int -\frac{(1-u)\sqrt{u}}{2}\d u.
\]
At this point, we are close to being done. Write
\begin{align*}
\int -\frac{(1-u)\sqrt{u}}{2}\d u &= \int \left(\frac{u\sqrt{u}}{2} - \frac{\sqrt{u}}{2}\right) \d u \\
&= \int \frac{u^{3/2}}{2} \d u - \int \frac{\sqrt{u}}{2} \d u \\
&= \frac{u^{5/2}}{5} - \frac{u^{3/2}}{3}.
\end{align*}
Now recall that $u = 1-x^2$. Hence our final answer is
\[
\int x^3\sqrt{1-x^2}\d x = \frac{(1-x^2)^{5/2}}{5} - \frac{(1-x^2)^{3/2}}{3}+C.
\]
\end{solution}

To summarize, if we suspect that a given function is the derivative of
another via the chain rule, we let $u$ denote a likely candidate for
the inner function, then translate the given function so that it is
written entirely in terms of $u$, with no $x$ remaining in the
expression. If we can integrate this new function of $u$, then the
antiderivative of the original function is obtained by replacing $u$
by the equivalent expression in $x$.



\begin{exercises}

\twocol

\begin{exercise} $\int (1-t)^9\d t$
\begin{answer} $-(1-t)^{10}/10+C$
\end{answer}\end{exercise}

\begin{exercise} $\int (x^2+1)^2\d x$
\begin{answer} $x^5/5+2x^3/3+x+C$
\end{answer}\end{exercise}

\begin{exercise} $\int x(x^2+1)^{100}\d x$
\begin{answer} $(x^2+1)^{101}/202+C$
\end{answer}\end{exercise}

\begin{exercise} $\int {1\over\root 3 \of {1-5t}}\d t$ 
\begin{answer} $-3(1-5t)^{2/3}/10+C$
\end{answer}\end{exercise}

\begin{exercise} $\int \sin^3x\cos x\d x$
\begin{answer} $(\sin^4x)/4+C$
\end{answer}\end{exercise}

\begin{exercise} $\int x\sqrt{100-x^2}\d x$
\begin{answer} $-(100-x^2)^{3/2}/3+C$
\end{answer}\end{exercise}

\begin{exercise} $\int {x^2\over\sqrt{1-x^3}}\d x$
\begin{answer} $-2\sqrt{1-x^3}/3+C$
\end{answer}\end{exercise}

\begin{exercise} $\int \cos(\pi t)\cos\bigl(\sin(\pi t)\bigr)\d t$
\begin{answer} $\sin(\sin\pi t)/\pi+C$
\end{answer}\end{exercise}

\begin{exercise} $\int {\sin x\over\cos^3 x}\d x$
\begin{answer} $1/(2\cos^2 x)=(1/2)\sec^2x+C$
\end{answer}\end{exercise}

\begin{exercise} $\int\tan x\d x$
\begin{answer} $-\ln|\cos x|+C$
\end{answer}\end{exercise}

\begin{exercise}  $\int_0^\pi\sin^5(3x)\cos(3x)\d x$
\begin{answer} $0$
\end{answer}\end{exercise}

\begin{exercise} $\int\sec^2x\tan x\d x$
\begin{answer} $\tan^2(x)/2+C$
\end{answer}\end{exercise}

\begin{exercise} $\int_0^{\sqrt{\pi}/2} x\sec^2(x^2)\tan(x^2)\d x$
\begin{answer} $1/4$
\end{answer}\end{exercise}

\begin{exercise} $\int {\sin(\tan x)\over\cos^2x}\d x$
\begin{answer} $-\cos(\tan x)+C$
\end{answer}\end{exercise}

\begin{exercise} $\int_3^4 {1\over(3x-7)^2}\d x$
\begin{answer} $1/10$
\end{answer}\end{exercise}

\begin{exercise} $\int_0^{\pi/6}(\cos^2x - \sin^2x)\d x$
\begin{answer} $\sqrt3/4$
\end{answer}\end{exercise}

\begin{exercise} $\int {6x\over(x^2 - 7)^{1/9}}\d x$
\begin{answer} $(27/8)(x^2-7)^{8/9}$
\end{answer}\end{exercise}

\begin{exercise} $\int_{-1}^1 (2x^3-1)(x^4-2x)^6\d x$
\begin{answer} $-(3^7+1)/14$
\end{answer}\end{exercise}

\begin{exercise} $\int_{-1}^1 \sin^7 x\d x$
\begin{answer} $0$
\end{answer}\end{exercise}

\begin{exercise} $\int f(x) f'(x)\d x$ 
\begin{answer} $f(x)^2/2$
\end{answer}\end{exercise}

\endtwocol

\end{exercises}










\section{Powers of Sine and Cosine}


Functions consisting of products of the sine and cosine can be
integrated by using substitution and trigonometric identities. These
can sometimes be tedious, but the technique is straightforward. The
basic idea in each case is to somehow take advantage of a
trigonometric identity, usually:
\[
\cos^2(x) + \sin^2(x) = 1, \qquad \sin^2(x) = \frac{1-\cos(2x)}{2}, \qquad \cos^2(x)= \frac{1+\cos(2x)}{2}.
\]
Some examples will suffice to explain the approach.

\begin{example}
Compute
\[\int \sin^5 x\d x.
\]
\end{example}

\begin{solution}
Rewrite the function:
$$
  \int \sin^5 x\d x=\int \sin x \sin^4 x\d x=
  \int \sin x (\sin^2 x)^2\d x=
  \int \sin x (1-\cos^2 x)^2\d x.
$$
Now use $u=\cos x$, $du=-\sin x\d x$:
\begin{align*}
  \int \sin x (1-\cos^2 x)^2\d x&=\int -(1-u^2)^2\d u \\
  &=\int -(1-2u^2+u^4)\d u \\
  &=-u+{2\over3}u^3-{1\over5}u^5+C \\
  &=-\cos x+{2\over3}\cos^3 x-{1\over5}\cos^5x+C. 
\end{align*}
\end{solution}

\begin{example}
Evaluate 
\[
\int \sin^6 x\d x.
\]
\end{example}

\begin{solution}
Use $\sin^2x =(1-\cos(2x))/2$ to rewrite the function:
\begin{align*}
  \int \sin^6 x\d x=\int (\sin^2 x)^3\d x&=
  \int {(1-\cos 2x)^3\over 8}\d x \\
  &={1\over 8}\int 1-3\cos 2x+3\cos^2 2x-\cos^3 2x\d x.
\end{align*}
Now we have four integrals to evaluate:
$$\int 1\d x=x$$
and
$$\int -3\cos 2x\d x = -{3\over 2}\sin 2x$$
are easy. The $\cos^3 2x$ integral is like the previous example:
\begin{align*}
  \int -\cos^3 2x\d x&=\int -\cos 2x\cos^2 2x\d x \\
  &=\int -\cos 2x(1-\sin^2 2x)\d x \\
  &=\int -{1\over 2}(1-u^2)\d u \\
  &=-{1\over 2}\left(u-{u^3\over 3}\right) \\
  &=-{1\over 2}\left(\sin 2x-{\sin^3 2x\over 3}\right).
\end{align*}
And finally we use another trigonometric identity,
$\cos^2x=(1+\cos(2x))/2$:
$$
  \int 3\cos^2 2x\d x=3\int {1+\cos 4x\over 2}\d x=
  {3\over 2}\left(x+{\sin 4x\over 4}\right).
$$
So at long last we get
$$
  \int \sin^6 x\d x = {x\over8} -{3\over 16}\sin 2x 
  -{1\over 16}\left(\sin 2x-{\sin^3 2x\over 3}\right)
  +{3\over 16}\left(x+{\sin 4x\over 4}\right)+C.
$$
\end{solution}

\begin{example}
Compute 
\[
\int \sin^2x\cos^2x\d x.
\]
\end{example}
\begin{solution} 
Use the formulas
$\sin^2x =(1-\cos(2x))/2$ and $\cos^2x =(1+\cos(2x))/2$ to get:
$$
  \int \sin^2x\cos^2x\d x=\int {1-\cos(2x)\over2}\cdot
  {1+\cos(2x)\over2}\d x.
$$
The remainder is left as an exercise.
\end{solution}

\begin{exercises}

\noindent Find the antiderivatives.

\twocol

\begin{exercise} $\int \sin^2 x\d x$
\begin{answer} $x/2-\sin(2x)/4+C$
\end{answer}\end{exercise}

\begin{exercise} $\int \sin^3 x\d x$
\begin{answer} $-\cos x+(\cos^3x)/3+C$
\end{answer}\end{exercise}

\begin{exercise} $\int \sin^4 x\d x$
\begin{answer} $3x/8-(\sin 2x)/4+(\sin 4x)/32+C$
\end{answer}\end{exercise}

\begin{exercise} $\int \cos^2 x\sin^3 x\d x$
\begin{answer} $(\cos^5 x)/5-(\cos^3x)/3+C$
\end{answer}\end{exercise}

\begin{exercise} $\int \cos^3 x\d x$
\begin{answer} $\sin x-(\sin^3x)/3+C$
\end{answer}\end{exercise}

\begin{exercise} $\int \sin^2 x\cos^2 x\d x$
\begin{answer} $x/8-(\sin 4x)/32+C$
\end{answer}\end{exercise}

\begin{exercise} $\int \cos^3 x \sin^2 x\d x$
\begin{answer} $(\sin^3x)/3-(\sin^5x)/5+C$
\end{answer}\end{exercise}

\begin{exercise} $\int \sin x (\cos x)^{3/2}\d x$
\begin{answer} $-2(\cos x)^{5/2}/5+C$
\end{answer}\end{exercise}

\begin{exercise} $\int \sec^2 x\csc^2 x\d x$
\begin{answer} $\tan x-\cot x+C$
\end{answer}\end{exercise}

\begin{exercise} $\int \tan^3x \sec x\d x$
\begin{answer} $(\sec^3x)/3-\sec x+C$
\end{answer}\end{exercise}

\endtwocol

\end{exercises}













\section{Integration by Parts}

While integration by substitution allows us to identify and ``undo''
the chain rule, \textit{integration by parts} allows us to recognize
the product rule.

\begin{mainTheorem}[Integration by Parts Formula] 
If $f(x)g(x)$ is differentiable on the interval $[a,b]$, then
\[
\int_a^b f(x) g'(x) \d x =f(x)g(x) \bigg|_a^b - \int_a^b f'(x) g(x) \d x.
\]
\end{mainTheorem}
\begin{proof} First note by the product rule we have
\[
\ddx f(x)g(x) = f(x)g'(x) + f'(x) g(x).
\]
Now integrate both sides of the equation above
\[
\int_a^b \ddx f(x) g(x) \d x = \int_a^b \left(f(x)g'(x) + f'(x) g(x)\right) \d x.
\]
By the Fundamental Theorem of Calculus, the left-hand side of the equation is
\[
f(x)g(x) \bigg|_a^b.
\]
However, by properties of integrals the right-hand side is equal to 
\[
\int_a^b f(x)g'(x)\d x + \int_a^b f'(x) g(x) \d x.
\]
Hence
\[
f(x)g(x) \bigg|_a^b = \int_a^b f(x)g'(x)\d x + \int_a^b f'(x) g(x) \d x.
\]
and so 
\[
 \int_a^b f(x)g'(x)\d x = f(x)g(x) \bigg|_a^b -  \int_a^b f'(x) g(x) \d x.
\]
\end{proof}


Integration by parts is often written in a more compact form
\[
\int u\d v = uv-\int v\d u,
\]
where $u=f(x)$, $v=g(x)$, $du=f'(x)\d x$ and $dv=g'(x)\d x$.  To use
this technique we need to identify likely candidates for $u=f(x)$ and
$dv=g'(x)\d x$.



\begin{example}
Compute
\[
\int \ln(x)\d x.
\]
\end{example}

\begin{solution}
Let $u=\ln(x)$ so $du=1/x\d x$. Hence,  $dv=1\d x$ so $v=x$
and so 
\begin{align*}
 \int \ln(x)\d x&=x\ln (x)-\int \frac{x}{x}\d x\\
&= x\ln (x)- x+C.\\
\end{align*}
\end{solution}

\begin{example}
Compute
\[
\int x\sin(x) \d x.
\]
\end{example}

\begin{solution} Let $u=x$ so $du=dx$. Hence, $dv=\sin(x)\d x$ so $v=-\cos(x)$ and
\begin{align*}
\int x\sin(x)\d x &=-x\cos(x)-\int -\cos(x)\d x\\
&= -x\cos(x)+\int \cos(x)\d x\\
&=-x\cos(x)+\sin x+C.
\end{align*}
\end{solution}


\begin{example}
Compute
\[
\int x^2\sin(x)\d x.
\] 
\end{example}

\begin{solution}
Let $u=x^2$, $dv=\sin(x)\d x$; then $du=2x\d x$ and $v=-\cos(x)$. 
Now 
\[
\int x^2\sin(x)\d x=-x^2\cos(x)+\int 2x\cos(x)\d x.
\] 
This is better than the original integral, but we need to do
integration by parts again. Let $u=2x$, $dv=\cos(x)\d x$; then $du=2$
and $v=\sin(x)$, and
\begin{align*}
  \int x^2\sin(x)\d x &=-x^2\cos(x)+\int 2x\cos(x)\d x \\
  &=-x^2\cos(x)+ 2x\sin(x) - \int 2\sin(x)\d x \\
  &=-x^2\cos(x)+ 2x\sin(x) + 2\cos(x) + C. 
\end{align*}
\end{solution}

Such repeated use of integration by parts is fairly common, but it can
be a bit tedious to accomplish, and it is easy to make
errors, especially sign errors involving the subtraction in the
formula. There is a nice tabular method to accomplish the calculation
that minimizes the chance for error and speeds up the whole
process. We illustrate with the previous example. Here is the
table:
\[
\begin{array}{|c|c|c|}\hline
\text{sign} & u & dv \\ \hline \hline
 & x^2 & \sin(x) \\ \hline
- & 2x & -\cos(x) \\ \hline
  & 2  & -\sin(x) \\ \hline
- & 0  & \cos(x) \\ \hline
\end{array}
\qquad\text{or}\qquad
\begin{array}{|c|c|}\hline
u & dv \\ \hline\hline
x^2 & \sin(x) \\ \hline 
-2x & -\cos(x) \\\hline
2 & -\sin(x)\\\hline
0 & \cos(x)\\\hline
\end{array}
\]

To form the first table, we start with $u$ at the top of the second
column and repeatedly compute the derivative; starting with $dv$ at
the top of the third column, we repeatedly compute the
antiderivative. In the first column, we place a ``$-$'' in every
second row. To form the second table we combine the first and second
columns by ignoring the boundary; if you do this by hand, you may
simply start with two columns and add a ``$-$'' to every second row.

To compute with this second table we begin at the top. Multiply the
first entry in column $u$ by the second entry in column $dv$ to get
$-x^2\cos(x)$, and add this to the integral of the product of the
second entry in column $u$ and second entry in column $dv$.  This
gives:
$$-x^2\cos(x)+\int 2x\cos(x)\d x,$$
or exactly the result of the first application of integration by
parts.  Since this integral is not yet easy, we return to the table.
Now we multiply twice on the diagonal, $(x^2)(-\cos(x))$ and
$(-2x)(-\sin(x))$ and then once straight across, $(2)(-\sin(x))$, and
combine these as
\[
-x^2\cos(x)+2x\sin(x)-\int 2\sin(x)\d x,
\]
giving the same result as the second application of integration by
parts. While this integral is easy, we may return yet once more to the
table. Now multiply three times on the diagonal to get
$(x^2)(-\cos(x))$, $(-2x)(-\sin(x))$, and $(2)(\cos(x))$, and once
straight across, $(0)(\cos(x))$. We combine these as before to get
\[
  -x^2\cos(x)+2x\sin(x) +2\cos(x)+\int 0\d x=
  -x^2\cos(x)+2x\sin(x) +2\cos(x)+C.
\]
Typically we would fill in the table one line at a time, until the
``straight across'' multiplication gives an easy integral. If we can
see that the $u$ column will eventually become zero, we can instead
fill in the whole table; computing the products as indicated will then
give the entire integral, including the ``$+C$'', as above.

\begin{exercises}
\noindent Compute the indefinite integrals.

\twocol

\begin{exercise} $\int x\cos x\d x$
\begin{answer} $\cos x+x\sin x+C$
\end{answer}\end{exercise}

\begin{exercise} $\int x^2\cos x\d x$
\begin{answer} $x^2\sin x-2 \sin x+2x\cos x +C$
\end{answer}\end{exercise}

\begin{exercise} $\int xe^x\d x$
\begin{answer} $(x-1)e^x +C$
\end{answer}\end{exercise}

\begin{exercise} $\int xe^{x^2}\d x$
\begin{answer} $(1/2)e^{x^2} +C$
\end{answer}\end{exercise}

\begin{exercise} $\int \sin^2 x\d x$
\begin{answer} $(x/2)-\sin(2x)/4 +C$
\end{answer}\end{exercise}

\begin{exercise} $\int \ln x\d x$
\begin{answer} $x\ln x-x +C$
\end{answer}\end{exercise}

\begin{exercise} $\int x\arctan x\d x$
\begin{answer} $(x^2\arctan x +\arctan x -x)/2+C$
\end{answer}\end{exercise}

\begin{exercise} $\int x^3\sin x\d x$
\begin{answer} $-x^3\cos x+3x^2\sin x+6x\cos x-6\sin x+C$
\end{answer}\end{exercise}

\begin{exercise} $\int x^3\cos x\d x$
\begin{answer} $x^3\sin x+3x^2\cos x-6x\sin x-6\cos x+C$
\end{answer}\end{exercise}

\begin{exercise} $\int x\sin^2 x\d x$
\begin{answer} $x^2/4-(\cos^2 x)/4-(x\sin x\cos x)/2+C$
\end{answer}\end{exercise}

\begin{exercise} $\int x\sin x\cos x\d x$
\begin{answer} $x/4-(x\cos^2 x)/2+(\cos x\sin x)/4+C$
\end{answer}\end{exercise}

\begin{exercise} $\int \arctan(\sqrt x)\d x$
\begin{answer} $x\arctan(\sqrt x)+\arctan(\sqrt x)-\sqrt{x}+C$
\end{answer}\end{exercise}

\begin{exercise} $\int \sin(\sqrt x)\d x$
\begin{answer} $2\sin(\sqrt x)-2\sqrt x\cos(\sqrt x)+C$
\end{answer}\end{exercise}

\begin{exercise} $\int\sec^2 x\csc^2 x\d x$
\begin{answer} $\sec x\csc x-2\cot x+C$
\end{answer}\end{exercise}

\endtwocol

\end{exercises}
